\documentclass[10pt,letterpaper]{article}
\usepackage[margin=0.5in]{geometry}

\usepackage{titlesec}
\usepackage{enumitem}
\usepackage{hyperref}
\usepackage[normalem]{ulem}   % allows underlining

% \usepackage[hidelinks]{hyperref}
\usepackage{parskip}
\usepackage{xcolor}
\usepackage{arimo}

\renewcommand{\familydefault}{\sfdefault}

\setlength{\parindent}{0pt}
\setlength{\parskip}{3pt}

\hypersetup{
  colorlinks=true,
  urlcolor=blue,
  linkcolor=blue
}

\renewcommand{\ULdepth}{1.8pt}  % makes underline look heavier
\renewcommand{\UrlFont}{\ul}    % applies underline to UR







\titleformat{\section}{\bfseries\large}{\thesection}{0.5em}{}
\titlespacing*{\section}{0pt}{4pt}{2pt}

\begin{document}

\begin{center}
  {\Large \textbf{Kathryn (Katy) Monopoli, PhD}}\\[4pt]
  \href{tel:+15083679077}{+1 508 367 9077} $\mid$ \href{mailto:krmonopoli@gmail.com}{krmonopoli@gmail.com} $\mid$ \href{https://www.linkedin.com/in/kathrynmonopoli}{linkedin.com/in/kathrynmonopoli} $\mid$ 
  \href{https://kmonopoli.github.io/}{\underline{kmonopoli.github.io}}
\end{center}

\vspace{4pt}

\section*{PROFESSIONAL SUMMARY}
Machine learning scientist with 10+ years of programming experience and expertise in developing predictive models for drug discovery and therapeutic development. Proven track record building production ready ML pipelines that directly impact pharmaceutical programs, with 2{,}000+ therapeutic designs and 34 molecules advancing to drug discovery. Strong combination of technical depth, biological domain knowledge, and cross functional collaboration skills. Experienced at bridging computational and experimental teams to deliver actionable insights from complex datasets.

\section*{CORE COMPETENCIES}
\textbf{Machine Learning and AI:}
Python (10+ yrs), TensorFlow, PyTorch, Scikit learn, XGBoost, deep learning, feature engineering, predictive modeling, model optimization, ensemble methods, cross validation, hyperparameter tuning

\textbf{Data Science:}
Statistical modeling, high dimensional data analysis, data pipelines, ETL, data cleaning and preprocessing, quality control, experimental design, A/B testing, model validation

\textbf{Drug Discovery:}
Therapeutic design, lead optimization, structure activity relationships, high throughput screening analysis, target selection, dose response modeling, translational research

\textbf{Technical Infrastructure:}
AWS, Docker, Git, SQL, MySQL, Bash, HPC, cloud computing, CI/CD, database design

\textbf{Domain Expertise:}
RNA therapeutics, genomics, transcriptomics, NGS data analysis, protein biophysics, antibody discovery

\textbf{Soft Skills:}
Project management, cross functional collaboration, technical communication, team leadership, mentoring

\section*{PROFESSIONAL EXPERIENCE}

\textbf{Graduate Researcher, Computational Biology and ML} \hfill 2019--2025\\
UMass Chan Medical School, Worcester MA

\textit{Machine Learning and Drug Discovery}
\begin{itemize}[leftmargin=*, itemsep=1pt, topsep=2pt]
  \item Developed and deployed AI and ML models enabling design of 2{,}000+ therapeutic molecules with 34 advancing to active drug discovery programs; models used across multiple therapeutic areas and disease targets.
  \item Pioneered a feature engineering approach using deep learning embeddings to capture complex sequence relationships, achieving a 5 fold improvement in predictive accuracy over baseline models.
  \item Built production ready ML pipelines for high throughput experimental data (NGS and screening assays) with automated QC, feature extraction, model training, and validation to ensure reproducibility and scalability.
  \item Applied diverse ML techniques including gradient boosting (XGBoost, LightGBM), ensemble methods, regularized linear models, SVMs, and neural networks with rigorous cross validation strategies.
  \item Optimized models for limited data scenarios by developing data partitioning and training strategies that enable predictive modeling on small biological datasets typical of early stage drug discovery.
\end{itemize}

\textit{Project Leadership and Collaboration}
\begin{itemize}[leftmargin=*, itemsep=1pt, topsep=2pt]
  \item Led software development as technical project manager, guiding an engineering team to integrate ML models into a production web application and coordinate timelines with experimental and computational stakeholders.
  \item Collaborated with medicinal chemists, biologists, and clinicians to define requirements, validate models, and translate predictions into experimental designs and go or no go decisions.
  \item Contributed to 4 patent applications and multiple high impact publications, including a manuscript in preparation for \textit{Nature Biotechnology}.
\end{itemize}

\textit{Data Science and Analysis}
\begin{itemize}[leftmargin=*, itemsep=1pt, topsep=2pt]
  \item Designed analysis frameworks for statistical modeling, data cleaning, normalization, batch correction, and parameter optimization across diverse experimental platforms.
  \item Created publication quality visualizations that communicate complex high dimensional relationships to both technical and non technical audiences.
  \item Developed database schemas and APIs for data management, retrieval, and integration across research programs.
\end{itemize}

\vspace{2pt}

\textbf{Associate Scientist, Bioinformatics} \hfill 2016--2018\\
Advirna (Biotech startup), Cambridge MA
\begin{itemize}[leftmargin=*, itemsep=1pt, topsep=2pt]
  \item Improved predictive algorithms for therapeutic design, increasing design throughput 5 fold while improving accuracy.
  \item Trained interpretable ML models optimized for small datasets using regularized approaches that balance performance and explainability for stakeholder trust.
  \item Developed and deployed a cloud based MySQL database platform to streamline data organization, classification, and access across R and D teams while minimizing operational costs.
\end{itemize}

\vspace{2pt}

\textbf{Researcher, Computational Biophysics} \hfill 2016\\
Institute for Protein Design (Baker Lab), Seattle WA
\begin{itemize}[leftmargin=*, itemsep=1pt, topsep=2pt]
  \item Implemented software components for the Rosetta protein folding suite in C++.
  \item Executed large scale molecular modeling campaigns using high throughput cloud computing infrastructure for protein design optimization.
\end{itemize}

\vspace{2pt}

\textbf{Intern, Antibody Discovery} \hfill 2013\\
Biogen, Cambridge MA
\begin{itemize}[leftmargin=*, itemsep=1pt, topsep=2pt]
  \item Characterized therapeutic monoclonal antibodies using biophysical assays (BLI, ELISA) for binding, stability, and functionality.
  \item Developed Python tools for antibody sequence analysis that accelerated epitope mapping and CDR analysis workflows.
\end{itemize}

\section*{EDUCATION}

\textbf{PhD, Computational Biosciences and Bioengineering} \hfill Nov 2025\\
UMass Chan Medical School and Worcester Polytechnic Institute\\
Advisors: Anastasia Khvorova (RNA therapeutics), Dmitry Korkin (Bioinformatics and CS)

\textbf{MS, Molecular and Cellular Biology} \hfill 2015\\
University of Massachusetts Amherst, Phi Beta Kappa, GPA 4.0

\textbf{BS, Biochemistry and Molecular Biology} \hfill 2014\\
University of Massachusetts Amherst, Honors College, \textit{summa cum laude}, GPA 4.0

\section*{TECHNICAL SKILLS}

\textbf{Languages:} Python, R, Java, C++, SQL, Bash\\
\textbf{ML and AI Frameworks:} TensorFlow, PyTorch, Scikit learn, XGBoost, LightGBM, Keras\\
\textbf{Data Science:} Pandas, NumPy, SciPy, Statsmodels, Jupyter\\
\textbf{Visualization:} Matplotlib, Seaborn, Plotly, ggplot2, Altair\\
\textbf{Bioinformatics:} Biopython, pysam, HTSeq, DESeq2, Scanpy\\
\textbf{Infrastructure:} Linux and Unix, Docker, Git and GitHub, AWS (EC2 and S3), Slurm and HPC\\
\textbf{Databases:} MySQL, PostgreSQL, SQLite\\
\textbf{Other:} Experimental design, statistical analysis, API development, technical writing

\section*{PUBLICATIONS AND PATENTS}

\textbf{Selected Publications}
\begin{itemize}[leftmargin=*, itemsep=1pt, topsep=2pt]
  \item Monopoli KR, Sostek B, Gross K, Alterman J, Korkin D, Khvorova A. Applying advanced AI methods to define features of functional therapeutic siRNAs. In preparation for \textit{Nature Biotechnology}.
  \item Davis SM, Hildebrand S, Monopoli KR, et al. Systematic analysis of siRNA and mRNA features impacting fully chemically modified siRNA efficacy. \textit{Nucleic Acids Research}, 2025.
  \item Monopoli KR, Korkin D, Khvorova A. Asymmetric trichotomous data partitioning enables predictive ML models using limited siRNA efficacy datasets. \textit{Molecular Therapy: Nucleic Acids}, 2023.
\end{itemize}

Full publication list: \href{https://scholar.google.com/citations?user=fYt4EYcAAAAJ}{Google Scholar profile}

\textbf{Patents}
    \begin{itemize}[leftmargin=*, itemsep=1pt, topsep=2pt]
    \item Oligonucleotides for MAPT modulation (US 17/204,480)
    \item Oligonucleotides for SNCA modulation (US 17/204,483)
    \item Oligonucleotides for MSH3 modulation (US 63/012,603)
    \item Oligonucleotides for SARS-CoV-2 modulation (US 17/333,839)
\end{itemize}
    
\section*{LEADERSHIP AND PROFESSIONAL DEVELOPMENT}

\textbf{Software Development Project Manager} \hfill 2023--2025\\
UMass Chan Medical School
\begin{itemize}[leftmargin=*, itemsep=1pt, topsep=2pt]
  \item Managed the full product development lifecycle for an ML powered therapeutic design platform.
  \item Led technical recruitment, hiring, and onboarding; coordinated budgets and resources across teams.
  \item Mentored junior engineers and facilitated cross functional communication to ensure successful delivery.
\end{itemize}

\textbf{Conference Presentations}\\
Invited speaker at industry conferences (ASGCT, OTS) and academic meetings (ISMB, Cold Spring Harbor), presenting ML methods and drug discovery applications to diverse audiences.

\end{document}

